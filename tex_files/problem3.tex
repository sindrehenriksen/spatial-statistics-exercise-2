\section{}
\label{sec:problem3}
%%%%%%%%%%%%%%%%%%%%%%%%%%%%%%%%%%%%%%%%%%%%%%%%%%%%%%%%%%%%%%%%%%%%%%
In this exercise we consider the Neuman-Scott (Mother-Child) event RF model. Using this model the mother-model has a stationary Poisson RF with intensity $\lambda_M$. There exists $k^M$ mothers, and for each mother there exist a number of children distributed according to a pdf $p(k^c)$ around the mother position $\vect{X}_j^M$ and is positionally distributed according to the conditional pdf $p(\vect{x} \given \vect{x}_j^M)$. The child position is then accepted to ensure that $\vect{x} \in \mathrm{D}$. The set of samples generated from a Neuman-Scott event RF is represented by $\matr{X}_D^{NS}:\{\vect{x}_1,\vect{x}_2,...,\vect{x}_{k_D}\}:\{\vect{x}_{ji};j=1,2,...,k_D^M:i = 1,2,...,k_D^c\};\vect{x}_j,\vect{x}_{ji}\in D$ with $k_d = \sum_j k_{D_j}^c$ given by the distribution in Equation \ref{eq:neumanscott}.
\begin{equation}
    \begin{array}{lll}
        X_D^{NS} & \sim & p(\vect{x}_{ji}; j = 1,2,...,k_D^M; i =1,2,...,k_{D_j}^c) \\
         & = & \prod\limits_{j=1}^{k_D^M}\left[\prod\limits_{i=1}^{k_{D_j}}\frac{ \vect{I}(\vect{x}_{ji}\in \mathrm{D})p(\vect{x}_{ji}\given\vect{x}_j^M)}{ P(\mathrm{D}\given\vect{x}_j^M)}\right]\times p(k_{D_j}^c\given k_j^c,\vect{x}_j^M)p(k_j^c)\times \frac{p(k_D^M)}{|D|^{k_D^M}} \\
         & = & \prod\limits_{j=1}^{k_D^M}\left[\prod\limits_{i=1}^{k_{D_j}}\frac{ \vect{I}(\vect{x}_{ji}\in \mathrm{D}) \phi_2(\vect{x}_{ji};\vect{x}_j^M,\sigma_c^2\matr{I}_2)}{ \Phi_2(D;\vect{X}_j^M,\sigma_c^2\matr{I}_2 )}\right] \\
         & \times & \Phi_2(D;\vect{x}_j^M,\sigma_c^2\matr{I}_2)^{k_{D_j}}[1-\Phi_2(\mathrm{D};\vect{x}_j^M,\sigma_c^2\matr{I}_2)]^{k_j^c-k_{D_j}^c}p(k_j^c)\\
          & \times & \frac{[\lambda_M]^{k_D^M}}{k_D^M!}exp\{-\lambda_M|D|\};k_{D_j}^c\leq k_j^c
         \end{array}
    \label{eq:neumanscott}
\end{equation}
From this we can see that the child position model is given by $\vect{x} \sim \Phi(\mathrm{D};\vect{x}_j^M, \sigma_c^2\matr{I}_2)$. Then the model parameters are $\vect{\theta}_{pNS} = \left[\lambda_M,\sigma_c^2,p(\cdot)\right]$, where $\lambda_M$ is the intensity of the number of mothers, $sigma_c^2$ is the deviation of the distance a child is from the mother, and $p(\cdot)$ is the distribution of the number of children around a mother. 

To find the parameter $\lambda_M$ a cluster analysis was done. Testing the redwood tree data against $30$ different algorithm for determining number of clusters in our data, or in other words number of mothers. Then a majority vote of the $30$ was thrown and the optimal number was $3$. The result of this is shown in Figure \ref{fig:numb_clust}, which gives us $\lambda_M = 3$. 
\begin{figure}
    \centering
    \includegraphics[scale=0.95]{figures/numb_clusters.pdf}
    \caption{}
    \label{fig:numb_clust}
\end{figure}
To further analyse the redwood three data a \textit{kmeans}-clustering was ran to partition the data points as children in $3$ clusters(mothers) found previously. The results is shown in Figure \ref{fig:cluster_part}. 
\begin{figure}
    \centering
    \includegraphics[scale=0.95]{figures/redwood_cluster_partitioning.pdf}
    \caption{}
    \label{fig:cluster_part}
\end{figure}
As for the third parameter $p(\cdot)$, the distribution of number of children in a cluster, was chosen to be a Poisson distribution. This means that the parameter of interest is the intensity $\lambda_C$. After the partitioning was found, the number of children in each cluster was known. This gives $\lambda_c = \frac{\sum\limits_{j = 1} ^{k^M} k_j^c}{k^M}$, which is the mean of the number of children in each cluster. And as for the positionally distribution around the mother the parameter $\sigma_c^2$ is needed. This is given by the sum of squares within a cluster $SS_j; j\in 1,2,...,k^M$ and the number of children in a cluster $k_j^c$. Given by the equation $\sigma_c^2 = \frac{\sum\limits_{j = 1}^{k_j^M}ss_j/(k_j^c-1)}{k_D}$. Now we have all the model parameters and a simulation was ran to find one realization of the Neuman-Scott event RF, shown in the Figure \ref{fig:cluster_event_rf}.
\begin{figure}
    \centering
    \includegraphics[scale=0.95]{figures/cluster_event_rf.pdf}
    \caption{}
    \label{fig:cluster_event_rf}
\end{figure}
Then a Monte Carlo test was run on $100$ realizations from the Neuman-Scott event RF. First the L-interaction function was calculated by equation \ref{eq:L_function} for each realization. The result of which is shown in Figure \ref{fig:gen_ns_l}.

\begin{figure}
    \centering
    \includegraphics[scale=0.65]{figures/gen_ns_l.pdf}
    \caption{}
    \label{fig:gen_ns_l}
\end{figure}

From the L-interaction function for the realizations the $0.05$ and $0.95$ quantiles were calculated and compared to the L-interaction function of the redwood tree data. 

\begin{figure}
    \centering
    \includegraphics[scale=0.65]{figures/ns_quant1.pdf}
    \caption{}
    \label{fig:ns_quant1}
\end{figure}

\begin{figure}
    \centering
    \includegraphics[scale=0.65]{figures/ns_quant2.pdf}
    \caption{}
    \label{fig:ns_quant2}
\end{figure}