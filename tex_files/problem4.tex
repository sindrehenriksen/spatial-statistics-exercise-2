\section{}
\label{sec:problem4}
%%%%%%%%%%%%%%%%%%%%%%%%%%%%%%%%%%%%%%%%%%%%%%%%%%%%%%%%%%%%%%%%%%%%%%

On the biological cell data we want to model a repulsive event RF with a Single-site McMC simulation of a Strauss Event RF. This model is specified on conditional form, given the event count $k_\textrm{D} = k$, realizations can be found by
%

\begin{equation}
    \begin{array}{rcl}
        \left[\mathbb{X}_\textrm{D}^S \given k_\textrm{D} = k\right] & \sim & p(\vect{x}_1,\vect{x}_2,...,\vect{x}_k \given k_\textrm{D} = k) \\
         & = & \mathrm{const} \times \prod\limits_{i,j\in \{1,2,...k\}} exp\{-\phi(\vect{x}_i-\vect{x}_j)\}
    \end{array}.
    \label{eq:straureal}
\end{equation}
%

Then let the distance between points in the domain be $\tau_{ij} = |\vect{\tau}_{ij}| \in \reals_\oplus; \vect{\tau}_{ij} = \vect{x}_i - \vect{x}_j$, where $\phi(\tau)$ is given by equation 
\begin{equation}
    \phi(\tau) = \begin{cases}
                    \phi_0; & 0 \leq \tau \leq \tau_0\\
                    \phi_0 exp\{-\phi_1[\tau-\tau_0]\}; & \tau > \tau_0
                \end{cases},
    \label{eq:interfunction}
\end{equation}
where $\tau_0 \in R_\oplus$ and $\phi_0,\phi_1 \in R_+$. This gives the model parameters for the Strauss-model $\vect{\theta} _{pS\given k} = [\tau_0,\phi_0,\phi_1]$. The minimum distance between the points in the cells data, $\tau_0$, is calculated by $\tau_0 = \min\limits_{\{i,j\in k;i\neq j\}} |\vect{x}_i-\vect{x}_j|$. The other parameters are chosen to fit L-interaction function of the realizations from the Strauss Event RF to the quantiles of the L-interaction function of the cells data. By Equation \eqref{eq:straureal} and Equation \eqref{eq:interfunction} a Single-site McMC algorithm is made and presented in Algorithm \ref{alg:SSS}. 

\begin{algorithm}[H]
\SetAlgoLined
initiate:\\
$\mathbb{X}_\textrm{D}^0\{\vect{x}_1^0,\vect{x}_2^0,...,\vect{x}_k^0\}; s.t. p(\vect{x}_1^0,\vect{x}_2^0,...,\vect{x}_k^0\given k_\textrm{D} = k)>0$\\
\For{$j = 1,2,...,k^M$}{
$\mathbb{X}_\textrm{D} = \mathbb{X}_\textrm{D}^{i-1}$\\
Generate:\\
$u\sim Uni[1,2,...,k]$\\
$\vect{x}^p \sim Uni[\textrm{D}]$\\
$\alpha = min\{1,exp\{-\sum\limits_{{j = 1};{j\neq u}}^k[\phi(\vect{x}^p - \vect{x}_j^i)-\phi(\vect{x}_u^i-\vect{x}_j^i)]\}\}$\\
\If{$Uni[0,1]< \alpha$}{
    $\mathbb{X}_\textrm{D}^i:\{\vect{x}_1^i,...,\vect{x}_{u-1}^i,\vect{x}^p,\vect{x}_{u+1}^i,...,\vect{x}_k^i\}$\\
}
}
\caption{Single-site McMC Simulation - Strauss Event RF}
 \label{alg:SSS}
\end{algorithm} 

An initial guess on the model parameter is done, then compared and changed until a fit is made. This yields the parameters  $\phi_0 = 7$ and $\phi_1 = 90$ and the minimum distance was calculated to be $\tau_0 = 0.0836$. Using these parameters and Algorithm \ref{alg:SSS} realizations is generated and a trace plot of the minimum distance between all the generated points in the simulation is presented in Figure \ref{fig:repulsive_trace}. In the trace a convergence towards the minimum distance in the cell data (the red line) is present and expected by the semi-hardcore model. The trace looks pretty bad for a McMC and it seems like the model is exploring little of the domain. Though this might be the case, the fact that a Single-site McMC is used and that the trace is of the minimum distance, its behavior can be explained. The minimum distance might not be update in each iteration, based on the fact that the point looked at, is chosen at random. The minimum distance can then end up not being updated for many iterations. In Figure \ref{fig:repulsive_event_rf} one realization of the Strauss event RF is shown, and a repulsive effect seems to be presented.  

The comparison between the L-interaction functions of the quantiles of the cell data and $100$ realizations from the Strauss Event RF is shown in Figure \ref{fig:gen_strauss_l}, Figure \ref{fig:strauss_quant1} and Figure \ref{fig:strauss_quant2}. 


\begin{figure}
    \centering
    \includegraphics[scale=0.9]{figures/repulsive_trace_rf.pdf}
    \caption{Trace plot of the minimum distance between all the points in an the McMC simulation of a Strauss Event RF with model parameters $[\tau_0,\phi_0,\phi_1] = [0.84,7,90]$. The red line is $\tau_0$, the minimum distance in the biological cell data.}
    \label{fig:repulsive_trace}
\end{figure}

\begin{figure}
    \centering
    \includegraphics[scale=0.9]{figures/repulsive_event_rf.pdf}
    \caption{One realization of a Single-site McMC simulation of a Strauss Event RF with model parameters $[\tau_0,\phi_0,\phi_1] = [0.84,7,90]$.}
    \label{fig:repulsive_event_rf}
\end{figure}

\begin{figure}
    \centering
    \includegraphics[scale=0.9]{figures/gen_strauss_l.pdf}
    \caption{L-interaction function of $100$ realizations of a Single-site McMC simulation of a Strauss Event RF with model parameters $[\tau_0,\phi_0,\phi_1] = [0.84,7,90]$. The black lines is the $0.05$ and $0.95$ quantile of the L-interaction function.}
    \label{fig:gen_strauss_l}
\end{figure}

\begin{figure}
    \centering
    \includegraphics[scale=0.9]{figures/strauss_quant1.pdf}
    \caption{A comparison between the $0.05$ and $0.95$ quantiles of the L-interaction function from $100$ realizations of a single-site McMC simulation of a Strauss Event RF with model parameters $[\tau_0,\phi_0,\phi_1] = [0.84,7,90]$, and the L-interaction function of the biological cell data.}
    \label{fig:strauss_quant1}
\end{figure}

\begin{figure}
    \centering
    \includegraphics[scale=0.9]{figures/strauss_quant2.pdf}
    \caption{A comparison between the $0.01$ and $0.99$ quantiles of the L-interaction function from $100$ realizations of a single-site McMC simulation of a Strauss Event RF with model parameters $[\tau_0,\phi_0,\phi_1] = [0.84,7,90]$, and the L-interaction function of the biological cell data.}
    \label{fig:strauss_quant2}
\end{figure}

In Figure \ref{fig:strauss_quant1} the L-function of the cells data follows the quantiles for the L-function of the Strauss realizations. From around $t>0.3$ they tend to diverge a little from each other, where the L-function for the cells data is a little higher. In the start at from $t=0$ to $t\approx 0.08$, which is around the value of $\tau_0$ the quantiles is $0$ this is, because we have a semi-hardcore model, where we restrict the minimum distance between the points in the realizations. Overall the Strauss model we implemented seems to fit the cells data good, there can still be improvements by choosing better model parameters, but it seems good enough.